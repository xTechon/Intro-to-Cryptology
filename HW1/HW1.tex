\documentclass[11pt]{article}
\usepackage{url}
\usepackage{amsmath} 
\usepackage{algorithm}
\usepackage{algpseudocodex}
\input{../macros.tex}

\newcommand{\bitsoracle}{{\mathsf{Bits}}}
\newcommand{\indcpa}{{\mathrm{IND}\mbox{-}\mathrm{CPA}}}
\newcommand{\indrcpa}{{\mathrm{IND}\$\mbox{-}\mathrm{CPA}}}
\newcommand{\Func}{\mathsf{Func}}
%\newcommand{\Perm}{\mathsf{Perm}}
\newcommand{\Prob}[1]{\Pr\left[#1\right]}
\newcommand{\calX}{\mathcal{X}}
\newcommand{\thatis}{i.e. }
\newcommand{\forexample}{e.g. }
 
%\newcommand{\Adv}{\mathrm{Adv}}
\newcommand{\ExpPRF}[2]{\mathsf{Exp}^{\mathrm{prf}}_{#1}{(#2)}}
\newcommand{\AdvPRF}[2]{\Adv^{\mathrm{prf}}_{#1} (#2)}
\newcommand{\ExpVP}[2]{\mathsf{Exp}^{\mathrm{vp}}_{#1}{(#2)}}
\newcommand{\AdvVP}[2]{\Adv^{\mathrm{vp}}_{#1} (#2)}
\newcommand{\ExpPRP}[2]{\mathsf{Exp}^{\mathrm{prp}}_{#1}{(#2)}}
\newcommand{\AdvPRP}[2]{\Adv^{\mathrm{prp}}_{#1} (#2)}
\newcommand{\ExpGuess}[2]{\mathsf{Exp}^{\mathrm{guess}}_{#1}{(#2)}}
\newcommand{\AdvGuess}[2]{\Adv^{\mathrm{guess}}_{#1} (#2)}
\newcommand{\ExpKR}[2]{\mathsf{Exp}^{\mathrm{kr-ic}}_{#1}{(#2)}}
\newcommand{\AdvKR}[2]{\Adv^{\mathrm{kr}}_{#1} (#2)}
%
\newcommand{\calV}{\mathcal{V}}

%%%%%%%%%%%%%%%%%%%%%%%%%%%%%%%%%%%%%%%%%%%%%%%%%%%%%%%%
\title{{\bf  Introduction to Modern Cryptography}\\ Problem Set 1 \\[2ex] }
\author{Daniel Hiromoto}
\begin{document}
  \maketitle
%%%%%%%%%%%%%%%%%%%%%%%%%%%%%%%%%%%%%%%%%%%%%%%%%%%%%%%%
  \thispagestyle{empty}

  \pagebreak{}
  
\begin{theorem}  Fix integers $n,k>0$ and Let $F\colon\bits^k \times \bits^n \to \bits^n$ be a blockcipher.  Define $E\colon \bits^{n+k} \times \bits^n \to \bits^n$ as $E_{K2 \concat K1}(X) = F_{K1}(X) \xor K2$.  Let~$A$ be a PRP-adversary (for attacking~$E$) that has time complexity~$t$ and asks at most~$q$ oracle queries.  Then there exists an adversary~$B$, explicitly given in the proof of this theorem, such that
  \begin{equation}
     \AdvPRP{E}{A} \leq \AdvPRP{F}{B} + \frac{1}{2^{n}!}
  \end{equation}
and where~$B$ has time complexity $t'=t + `negligable`$ and asks $q'=q+2$ queries. \hfill$\diamondsuit$
\end{theorem}


%%%%%%%%%%%%%%%%%%%%%%%%%%%%%%%%%%%


\begin{figure}[hb]
\center{}
\fpage{.7}{
\hpagess{.6}{.45}
{
\experimentv{$\ExpKR{E}{A}$:}\\
$K \getsr \bits^{2n}$\\
$F \getsr \mathsf{BC}(n,n)$\\
$K' \getsr A^{\calO(\cdot),\mathrm{IC}(\cdot,\cdot)}$\\
Ret $[K'=K]$\\
}
{
\oraclev{$\calO(X)$}\\
Ret $E_K(X)$\\

\oraclev{$\mathrm{IC}(L,X)$}\\
Ret $F_L(X)$
}
}
\caption{The key-recovery notion for function family $E\colon\bits^{2n}\times\bits^n \to \bits^n$, in the ideal cipher model for some (related) blockcipher~$F$.  In the current problem, $E$ is specifically defined as $E_{K2\concat K1}(X)=F_{K1}(X)\xor K2$, so a single call $\calO(X)$ effectively determines one value of the table for $F_{K1}$, in the lazy-sampling sense. (But \emph{only} for that specific table, not any of the other $F_{K1'}$ tables for $K1' \neq K1$.)}\label{fig:eks}
\end{figure}


%%%%%%%%%%%%%%%%%%%%%%%%%%%%%%%%%%%%%%%%%%


\paragraph{Proof} Let's define a notion where $k = n$, and create a function family 
$E\colon\bits^{2n}\times\bits^n \to \bits^n$ for some ideal cipher model with some experiment using adversary~$A$ (\figref{fig:eks}).
We can then create a new notion (\figref{fig:eprp}) as a superset of \figref{fig:eks} for some bit-guessing game with adversary~$B$. The advantage then, of adversary~$B$ would be:
\begin{equation}
  \AdvPRP{F}{B} = 2 \times \Pr[\ExpPRP{F}{B} = 1] - 1
\end{equation}


%%%%%%%%%%%%%%%%%%%%%%%%%%%%%%%%%%%%%%%%%%%


\begin{figure}[hb]
\center{}
\fpage{.7}{
\hpagess{.6}{.45}
{
\experimentv{$\ExpPRP{F}{B}$:}\\
$b \getsr \bits$ \\
$L \getsr \bits^{k}$\\
$V \getsr \bits^{n}$\\
$ \pi \getsr \mathsf{perm}(n)$\\
$L' \getsr B^{\calO(\cdot),\mathrm{F}(\cdot, \cdot)}$\\
Ret $[L'=L]$\\
}
{
\oraclev{$\calO(Y)$}\\
if $ b = 1 $\\
  then $Y \leftarrow E_{V \Vert L}(X)$ \\
  else $Y \leftarrow \pi(x) \xor V$\\
Ret $Y$\\

\oraclev{$\mathrm{F}(L, Y)$}\\
if $ b = 1 $\\
  then $Y \leftarrow F_L(X)$\\
  else $Y \leftarrow \pi(X)$\\
Ret $Y$
}
}
\caption{The Bit-Guessing Game notion for function family $E\colon\bits^{k+n}\times\bits^n \to \bits^n$, in the ideal cipher model for some (related) blockcipher~$F$.  
In the current problem, $E$ is specifically defined as $E_{V\concat L}(X)=F_{L}(X)\xor V$, so a single call $\calO(Y)$ effectively determines one value of the table for $F_{L}$ which may be from a PRP, in the lazy-sampling sense. 
(But \emph{only} for that specific table, not any of the other $F_{L'}$ tables for $L' \neq L$.)}\label{fig:eprp}
\end{figure}

%%%%%%%%%%%%%%%%%%%%%%%%%%%%%%%%%%%

The $\Pr[\ExpPRP{F}{B} = 1]$ (probability of B winning the game) can then be expanded as:
\begin{equation}
  \begin{aligned}
  \Pr[\ExpPRP{F}{B} = 1] 
                        & = \Pr(b = 1) \times \Pr[\ExpPRP{F}{B} = 1 \vert b = 1]  \\
                        & + \Pr(b = 0) \times \Pr[\ExpPRP{F}{B} = 1 \vert b = 0] \\
                        \\
                        & = \frac{1}{2} \times \Pr[\ExpPRP{F}{B} = 1 \vert b = 1]  \\
                        & + \frac{1}{2} \times \Pr[\ExpPRP{F}{B} = 1 \vert b = 0] 
  \end{aligned}
\end{equation}


%%%%%%%%%%%%%%%%%%%%%%%%%%%%%%%%%%%%%%%%%%


If we define adversary~$B$'s behavior as (Algorithm~\ref{alg:advb}),
then $\Pr[\ExpPRP{F}{B} = 1 \vert b = 1]$ can only be true if and only if $b = 1$ and at least as often as 
adversary~$A$ wins its game ($A^{\mathrm{F_{K_1}}(X) \xor K_2} => K1 \concat K2$).


%%%%%%%%%%%%%%%%%%%%%%%%%%%%%%%%%%%%%%%%%%


\algrenewcommand\algorithmicif{\textbf{When}}
%\algrenewcommand\algorithmicfor{\textbf{Pick}}
\begin{algorithm}{}
  \caption{Adversary B}\label{alg:advb}
\begin{algorithmic}[1]
  \State{} Run $A$
  \If{$A$ querries $X$}
    \State{} $Y \gets \calO(Y)$
  \EndIf{}
  \If{$A$ querries $L,X$}
    \State{} $Y \gets \mathrm{F}(L,Y)$
  \EndIf{}
  \If{$A$ Halts with output $K' = k_{2} \concat k_{1}$}
    \For{$W$ not previously querried}
      \State{} $U \gets \mathrm{E_{K'}}(W)$
      \State{} $I \gets \mathrm{F_{k_1}}(W)$
      \State{} $V \gets \calO(W)$
      \State{} $J \gets \mathrm{F}(F_{k_1}, W)$
      \State{} \Return{} [$U = V$ \AND{} $I = J$]
    \EndFor{}
  \EndIf{}
  
\end{algorithmic}
\end{algorithm}


%%%%%%%%%%%%%%%%%%%%%%%%%%%%%%%%%%%%%%%%%%%


The $\Pr[\ExpPRP{F}{B} = 1 \vert b = 1]$ then, would be greater than $\Pr[\ExpKR{E}{A}]$.
For $\Pr[\ExpPRP{F}{B} = 1 \vert b = 0]$, can be simplified.
\begin{equation}
  \begin{aligned}
    \Pr[\ExpPRP{F}{B} = 1 \vert b = 0] 
                                    & = 1 - \Pr[\ExpPRP{F}{B} = 0 \vert b = 0] \\
                                    \\
    \Pr[\ExpPRP{F}{B} = 1 \vert b = 0] & = 1 - \frac{1}{2^{n}!}
  \end{aligned}
\end{equation}

We can simplify $\Pr[\ExpPRP{F}{B} = 0 \vert b = 0]$ by the following theorem:

\begin{theorem}
  For any fixed $V \in \bits^n$ and any $\pi \in \Perm{n}$, $f(X)=\pi(X) \xor V$ is a permutation; and for any 
  fixed $V \in \bits^n$ and $\pi'\in \Perm{n}$, when $\pi \getsr \Perm{n}$ we have $\Prob{f=\pi'}=1/(2^n!)$
\end{theorem}

As $\Pr[\ExpPRP{F}{B} = 0 \vert b = 0]$ will use the PRP, we will simplify it to $\frac{1}{2^{n}!}$. Now, we can combine our equations as:

\begin{equation}
  \begin{aligned}
  \Pr[\ExpPRP{F}{B} = 1] 
                        & = \Pr(b = 1) \times \Pr[\ExpPRP{F}{B} = 1 \vert b = 1]  \\
                        & + \Pr(b = 0) \times \Pr[\ExpPRP{F}{B} = 1 \vert b = 0] \\
                        \\
                        & = \frac{1}{2} \times \Pr[\ExpPRP{F}{B} = 1 \vert b = 1]  \\
                        & + \frac{1}{2} \times \Pr[\ExpPRP{F}{B} = 1 \vert b = 0] \\
                        \\
                        & \geq \frac{1}{2}(\AdvPRP{E}{A}) + \frac{1}{2}(1-\frac{1}{2^{n}!})
  \end{aligned}
\end{equation}

Using this in our original equation:

\begin{equation}
  \begin{aligned}
    \AdvPRP{F}{B} & = 2 \times \Pr[\ExpPRP{F}{B} = 1] - 1\\
                  & \geq 2 \times (\frac{1}{2}(\AdvPRP{E}{A}) + \frac{1}{2}(1-\frac{1}{2^{n}!})
) - 1 \\ 
                  & \geq \AdvPRP{E}{A} - \frac{1}{2^{n}!} \\
     \AdvPRP{E}{A} & \leq \AdvPRP{F}{B} + \frac{1}{2^{n}!}
  \end{aligned}
\end{equation}

As adversary~$B$ must run at least as long as and only make 2 more queries (Algorithm \ref{alg:advb} line 6) 
than adversary~$A$, the $t'$ and $q'$ of adversary~$B$ must be $t'=t + `negligable`$ and $q'=q+2$

\paragraph{Question 2} I ran out of time\ldots{}

\end{document}

